\section{Economic Analysis}\label{sec:economy}
\phantomsection

\subsection{Project description}
For the last several decades Information Technology has advanced like no other field in the history of humankind. The development of fast computers and optimized algorithms has opened a big field for medical appliances and analysis of medical data. Respectively big and small health problems can now be solved much easily. Sleep Apnea is a serious disorder, which should be addressed by everybody. Since the process of going through a polysomnographic test is a hideous technique, a lot of other ways of detecting Sleep Apnea are proposed. In a society where healthcare has one of the biggest priorities, it is reasonable to develop applications for detecting or solving health problems. Snoring issues are considered a nuisance and are gladly removed from one's life.

From this perspective is it clear why a snore and Sleep Apnea detecting application is a very useful and desired one. It can be used both in professional medical field by experts that want to automate and help their patients get comfortable with solutions that they offer, and additionally it is a perfect choice for those who suspect health problems and wish to identify the heaviness of snoring. 

As a cross-platform application it is flexible and easy to deploy on various platforms. Users of different ages with suspected snoring problems can run the application. It has a simple and intuitive interface. The application requires a device with a microphone -- this is so far the biggest requirement.

\subsection{Project time schedule}
For the accomplishment of a project it is necessary to establish a schedule. For the development of the snore detecting app, agile project management is applied to offer flexible and iterative method of designing the application. It goes in 5 stages: planning, research, development, testing and deployment. The steps are repetitive and incremental.

\subsubsection{Objective determination}
The main objective of the following project is to provide a complete and functioning application for snore detection. Otherwise without a finished product there is no profit. More to that, it is important to market the application and get exposed to a large audience in need. This can be done via medical institutions that can provide the software to their patients. Another way is to market oneself on platform-specific stores. Since it is not a common piece of software, it creates a very specific audience of users, the ones that have a real demand in snore recognition. 

The other crucial objective is to be as precise as possible. Since it deals with one's health, it becomes no joke. Even though the application states its status of being not responsible for the results and advices it gives, it is still a big influence on one's health decisions. 

To keep up with the latest trends and researches, it is also an essential objective to keep updated and provide enhancements to the software. The lifecycle of the application will require bugfixes, interface changes, feature implementations. All of that will help the system still be trendy and up-to-date on the market.

\subsubsection{Time schedule establishment}
As it was said above the project will iterate over 5 steps: planning, research, development, testing and deployment. Naturally as most of the IT projects, it is subdivided into smaller parts. Planning step isn't supposed take up a lot of time, since the requirements are flexible. Moreover due to the research part the design solutions can change over time and open up new perspectives. The process of development is being split up in smaller tasks that can be accomplished within a 2-5 day period. Total duration of the project is computed using Formula \eqref{eq:duration}.

\begin{equation} \label{eq:duration}
 Duration = FinishDate - StartDate + ReserveTime
\end{equation}

In table \ref{table:schedule} is presented the first iteration of the project schedule. It uses the following notations: PM -- project manager, SA -- system architect, SM -- sales manager, D -- developer.

\begin{table}[!ht]
\begin{center}
\caption{Time schedule}
\begin{tabularx}{\textwidth}{| M{1.5em} | Y | Y | Y | Y |N}
\hline
\textbf{Nr} & \textbf{Activity Name} & \textbf{Duration (days)} & \textbf{People involved} & \textbf{Comments} &\\[18pt]
\hline
1 & Define the project concept and objectives & 5 & PM, SA, SM, D & It is a common task &\\[14pt]
\hline
2 & Perform market analysis & 10 & PM, SA & Will results into a document describing market analysis &\\[14pt]
\hline
3 & Analysis of the domain & 15 & SA, D & Research of the recognition algorithms &\\[14pt]
\hline
4 & Write down requirements and specifications & 5 & PM, SA, D & &\\[14pt]
\hline
5 & System design (UML) & 10 & PM, SA, D & &\\[14pt]
\hline
6 & Database design & 3 & PM, SA, D & Development database and end-user database schemes &\\[14pt]
\hline
7 & Preprocessing and learning part of the implementation & 30 & PM, SA, D & &\\[14pt]
\hline
8 & End-user application development & 20 & PM, SA, D, SM & &\\[14pt]
\hline
9 & Validation of results & 10 & PM, SA, D, SM & &\\[14pt]
\hline
10 & Documentation & 5 & D & &\\[14pt]
\hline
11 & Deployment and testing & 10 & PM, SA, D & &\\[14pt]
\hline
12 & Active marketing & 5 & SM & &\\[14pt]
\hline
13 & Total time to finish the system & 128 & & &\\[14pt]
\hline
\end{tabularx}
\label{table:schedule}
\end{center}
\end{table}

Table \ref{table:schedule} describes the activities that will occur during project development, who is involved into each process and how much time does it take to accomplish a task. Total amount of time spent on the following project is 128 days.

\subsection{Project description}
For the last several decades Information Technology has advanced like no other field in the history of humankind. The development of fast computers and optimized algorithms has opened a big field for medical appliances and analysis of medical data. Respectively big and small health problems can now be solved much easily. Sleep Apnea is a serious disorder, which should be addressed by everybody. Since the process of going through a polysomnographic test is a hideous technique, a lot of other ways of detecting Sleep Apnea are proposed. In a society where healthcare has one of the biggest priorities, it is reasonable to develop applications for detecting or solving health problems. Snoring issues are considered a nuisance and are gladly removed from one's life.

From this perspective is it clear why a snore and Sleep Apnea detecting application is a very useful and desired one. It can be used both in professional medical field by experts that want to automate and help their patients get comfortable with solutions that they offer, and additionally it is a perfect choice for those who suspect health problems and wish to identify the heaviness of snoring. 

As a cross-platform application it is flexible and easy to deploy on various platforms. Users of different ages with suspected snoring problems can run the application. It has a simple and intuitive interface. The application requires a device with a microphone -- this is so far the biggest requirement.

\subsection{Economic motivation}
The following section describes the evaluation of the project from the economic point of view. That includes the total profit, number of potential clients, salaries that have to be paid to employees, revenues that the company gets by commercializing the product. All the costs and prices are given in MDL (Moldavian lei) currency. Tangible and intangible assets, indirect expenses will also be taken into account. Wear and depression in regard to final product will also be computed.

\subsubsection{Tangible and intangible asset expenses}
The budget for the required assets is shown in Table \ref{table:tangible_assets}, Table \ref{table:intangible_assets}, Table \ref{table:direct_expenses}.

\begin{table}[!ht]
\begin{center}
\caption{Tangible asset expenses}
\begin{tabularx}{\textwidth}{| M{5em} | Y | M{7em} | M{5em} | M{5em} | M{5em} |N}
\hline
\textbf{Material} & \textbf{Specification} & \textbf{Measurement unit} & \textbf{Price per unit (MDL)} & \textbf{Quantity} & \textbf{Sum (MDL)} &\\[18pt]
\hline
Computer & Lenovo X201 i5 processor & Unit & 10000 & 1 & 10000 &\\[14pt]
\hline
Smartphone & Nexus 5 & Unit & 8500 & 1 & 8500 &\\[14pt]
\hline
Total & & & & & 18500 &\\[14pt]
\hline
\end{tabularx}
\label{table:tangible_assets}
\end{center}
\end{table}

\begin{table}[!ht]
\begin{center}
\caption{Intangible asset expenses}
\begin{tabularx}{\textwidth}{| M{5em} | Y | M{7em} | M{5em} | M{5em} | M{5em} |N}
\hline
\textbf{Material} & \textbf{Specification} & \textbf{Measurement unit} & \textbf{Price per unit (MDL)} & \textbf{Quantity} & \textbf{Sum (MDL)} &\\[18pt]
\hline
License & Enterprise Architect Desktop Edition License & Unit & 1900 & 3 & 5700 &\\[14pt]
\hline
License & PyCharm Commercial License & Unit & 2800 & 3 & 8400 &\\[14pt]
\hline
Total & & & & & 14100 &\\[14pt]
\hline
\end{tabularx}
\label{table:intangible_assets}
\end{center}
\end{table}

\begin{table}[!ht]
\begin{center}
\caption{Direct expenses}
\begin{tabularx}{\textwidth}{| M{5em} | Y | M{7em} | M{5em} | M{5em} | M{5em} |N}
\hline
\textbf{Material} & \textbf{Specification} & \textbf{Measurement unit} & \textbf{Price per unit (MDL)} & \textbf{Quantity} & \textbf{Sum (MDL)} &\\[18pt]
\hline
Single-board computer & Raspberry Pi Model B  & Unit & 600 & 1 & 600 &\\[14pt]
\hline
USB microphone & Logitech Labtec Microphone & Unit & 600 & 1 & 600 &\\[14pt]
\hline
Whiteboard & Universal Dry Erase Board & Unit & 500 & 1 & 500 &\\[14pt]
\hline
Paper & A4 & 500 sheets & 60 & 1 & 60 &\\[14pt]
\hline
Marker & Whiteboard marker & Unit & 10 & 5 & 50 &\\[14pt]
\hline
Pen & Blue pen & Unit & 3 & 3 & 9 &\\[14pt]
\hline
Total & & & & & 1819 &\\[14pt]
\hline
\end{tabularx}
\label{table:direct_expenses}
\end{center}
\end{table}

So the total amount of direct expenses represents 

\begin{equation}
 18500 + 14100 + 1819 = 34419 lei
\end{equation}

\subsubsection{Salary expenses}
The employees obviously should be paid accordingly. Hence this section is concerned about the salaries to employees and various funds that should be paid. The distribution of salaries is the following: project manager -- 350 lei, system architect -- 450 lei, sales manager -- 300 lei, developer -- 420 lei. Those expenses are introduces in Table \ref{table:salaries}.

\begin{table}[!ht]
\begin{center}
\caption{Salary expenses}
\begin{tabularx}{\textwidth}{| Y | M{8em} | M{8em} | M{8em} |N}
\hline
\textbf{Employee} & \textbf{Work fund (days)} & \textbf{Salary per day (MDL)} & \textbf{Salary fund (MDL)}&\\[18pt]
\hline
Project Manager & 103 & 350 & 36050 &\\[14pt]
\hline 
System Architect & 108 & 450 & 48600 &\\[14pt]
\hline
Sales Manager & 40 & 300 & 12000 &\\[14pt]
\hline
Developer & 113 & 420 & 47460 &\\[14pt]
\hline
Total & & & 144110 &\\[14pt]
\hline
\end{tabularx}
\label{table:salaries}
\end{center}
\end{table}

Now by having computed all the salaries for the employees, it is time to compute how much to be paid to social services fund, medical insurance fund and the total work expenses by summing up all previous expenses. So the social service fund which is $23\%$ this year will be 

\begin{equation}
 FS = F_{re} \cdot T_{fs} = 144110 lei \cdot 23 \% = 33145.3 lei 	
\end{equation}

where $F_{re}$ is the salary expense fund and $T_{fs}$ is the social service tax approved each year.

The medical insurance fund is computed as

\begin{equation}
 MI = F_{re} \cdot T_{mi} = 144110 lei \cdot 4 \% = 5764.4 lei
\end{equation}

where $T_{mi}$ is the mandatory medical insurance tax approved each year by law of medical insurance and this year it is $3.5\%$. 

So now having computed social service tax and medical insurance tax, it is possible to compute total work expense fund as follows

\begin{equation}
 WEF = F_{re} + FS + MI = 144110 lei + 33145.3 lei + 5764.4 lei = 183019.7 lei
\end{equation}

where $WEF$ is the work expense fund, FS is the social fund and MI is the medical insurance fund. In that way the total work expense fund was computed. 

\subsection{Individual person salary}
Along with total work expense fund, it is necessary to compute the annual salary for the developer. Considering that the developer has a salary of 420 lei per day and there are totally 250 working days in the year, so the gross salary that the developer get is

\begin{equation}
 Gross Salary = 420 lei \cdot 250 days = 105000 lei
\end{equation}

Social fund tax this year represents $6\%$, so the amount that should be tax paid is

\begin{equation}
 Social Fund = 105000 lei \cdot 6\% = 6300 lei
\end{equation}

Medical insurance tax represents $4\%$ and gives the following result

\begin{equation}
 Medical Insurance Fund = 105000 lei \cdot 4\% = 4200 lei
\end{equation}

In order to proceed with income tax computations, it is necessary to calculate the amount of taxed salary.

\begin{equation}
\begin{split}
 Taxed Salary &= Gross Salary - Social Fund - Medical Insurance Fund - Personal Exemption \\
              &= 105000 - 6300 - 4200 - 9516 = 84984 lei
\end{split}
\end{equation}

The last but not the least thing to be computed is the total income tax, which is $7\%$ for income under 27852 lei and $18\%$ for income over 27852 lei.

\begin{equation}
\begin{split}
 Income Tax &= Taxed Salary - Salary Tax \\
	    &= 27852 \cdot 7\% + (84984 - 27852) \cdot 18\% \\
	    & = 1949.64 + 10283.76 = 12233.4 lei
 \end{split}
\end{equation}

With all this now it is possible to find out what's going to be the net income.

\begin{equation}
\begin{split}
 Net Salary &= Gross Salary - Income Tax - Social fund - Medical Insurance Fund \\
            &= 105000 - 12233.4 - 6300 - 4200 = 82266.6 lei
\end{split}
\end{equation}

\subsubsection{Indirect expenses}
The indirect expenses are things like electricity, Internet traffic, water, etc. Those will be presented in Table \ref{table:indirect_expenses}.

\begin{table}[!ht]
\begin{center}
\caption{Indirect expenses}
\begin{tabularx}{\textwidth}{| M{5em} | Y | M{7em} | M{5em} | M{5em} | M{5em} |N}
\hline
\textbf{Material} & \textbf{Specification} & \textbf{Measurement unit} & \textbf{Price per unit (MDL)} & \textbf{Quantity} & \textbf{Sum (MDL)} &\\[18pt]
\hline
Internet & Moldtelecom & Pack & 200.00 & 3 & 600.00 &\\[14pt]
\hline
Transport & Public bus & Trip & 3.00 & 132 & 396 &\\[14pt]
\hline
Phone & Moldtelecom & Pack & 30.00 & 3 & 90 &\\[14pt]
\hline
Electricity & Union Fenosa & KWh & 1.58 & 250 & 395 &\\[14pt]
\hline
Total & & & & & 1481 &\\[14pt]
\hline
\end{tabularx}
\label{table:indirect_expenses}
\end{center}
\end{table}

\subsubsection{Wear and depreciation}
Another important part of economic analysis is the computation of wear and depreciation. It is a well known fact that any product decreases its value with time. Depression will be computed uniformly for the whole project duration, so that there are no accountancy issues. In other words, if a product is planned for 3 years, it should be divided into 3 uniform parts according to each year. Straight line depreciation will be applied. Normally wear is computed regarding to the type of asset. The notebook and single-board computer are usable for a period of 3 years. Licenses will last for a single year. At first tangible and intangible assets are summed up and then the salvage costs of each of the items at the end of their period of use has to be subtracted:

\begin{equation}
 \begin{split}
  Total Asset Value &= \sum_{} (AssetCost - Salvage Value) \\
		    &= (10000 - 200) + (8500 - 200) + (5700 - 200) + (8400 - 200) \\
		    &= 31800 lei
 \end{split}
\end{equation}

In order to get the yearly wear, divide total asset value by the period of use of assets

\begin{equation} \label{eq:wear}
 \begin{split}
  Wear Per Year &= Total Asset Value / Period Of Use \\
                &= 31800 lei/3 years= 10600 lei per year
 \end{split}
\end{equation}

Formula \eqref{eq:wear} included tangible assets which will last for 3 years and intangible assets which last only one year. The initial value of assets was

\begin{equation}
 InitialAssetValue = Tangible Assets + Intangible Assets = 18500 + 14100 = 32600 lei
\end{equation}

The project takes 128 days to be finish, respectively the wear value of the assets in this particular case should be

\begin{equation}
 \begin{split}
  ProjectWearValue &= WearPerYear / DaysInYear \cdot ProjectDuration\\
                   &= 32600 lei / 365 days \cdot 128 days = 11432.32 lei
 \end{split}
\end{equation}

\subsubsection{Product cost}
With all the project expenses computed, it is easy to compute the product cost which includes direct and indirect expenses, salary expenses and wear expenses as shown in Table \ref{table:product_cost}.

\begin{table}[!ht]
\begin{center}
\caption{Total Product Cost}
\begin{tabularx}{\textwidth}{| Y | M{10em} | M{10em} |N}
\hline
\textbf{Expense type} & \textbf{Sum (MDL)} & \textbf{Percentage (\%)} &\\[18pt]
\hline
Direct expenses & 1819 & 1.07 &\\[14pt]
\hline
Indirect expenses & 1481 & 0.87 &\\[14pt]
\hline
Salary expenses & 144110 & 85.36  &\\[14pt]
\hline
Social fund expenses & 6300 & 3.73 &\\[14pt]
\hline
Medical insurance expenses & 4200 & 2.17 &\\[14pt] % fix here
\hline
Asset wear expenses & 11432 & 6.77 &\\[14pt]
\hline
\textbf{Total product cost} & \textbf{168817} & \textbf{100} &\\[14pt] 
\hline
\end{tabularx}
\label{table:product_cost}
\end{center}
\end{table}

\subsubsection{Economic indicators and results}
At this point it is crucial to sell the product to clients from medical institutions and end-users. The total product cost is very high, consequently there are 2 strategies that can be applied -- whether sell less with a high price or sell more with a lower price. It is not possible to add a percentage to the product cost that will represent the profit. It is assumed that the expected profit represents $15\%$ of the total product cost and the expected number of sold copies to be 1000. 

\begin{equation}
 \begin{split}
  Gross Price &= Total Product Cost / Number Of Copies Sold + Chosen Profit Percentage\\
              &= 168817 lei/1000 copies + 15\% = 194.13 lei
 \end{split}
\end{equation}

This is not the price of the end product, since it is necessary to add sales tax (VAT), which represents $20\%$ and is added to the gross price. 

\begin{equation}
 \begin{split}
  Sale Price (incl. TVA) &= Gross Price + Sales Tax\\
              &= 194.13 lei + 20\% = 232.96 lei
 \end{split}
\end{equation}

The net income is computed by multiplying gross price and the number of expected copies to be sold, which will be

\begin{equation}
 \begin{split}
  Net Income &= Gross Price \cdot Number Of Copies Sold\\
              &= 194.13 lei \cdot 1000 = 194130 lei
 \end{split}
\end{equation}

Moreover it is necessary to compute the gross and net profit. Gross profit is computed by subtracting cost of production from the net income. Net profit can be found by subtracting $12\%$ (because it is a legal entity) from the gross profit. The indicators are the following

\begin{equation}
 \begin{split}
  Gross Profit &= Net Income - Cost Of Production\\
              &= 194130 - 168817 = 25313 lei\\              
  Net Profit &= Gross Profit - 12\% \\
             &= 25313 - 12\% = 22275.44 lei
 \end{split}
\end{equation}

The profitability indicators are computed below:

\begin{equation}
 \begin{split}
  Cost Profitability &= Gross Profit / Cost Of Production \cdot 100\%\\
              &= 25313 / 168817 \cdot 100\% = 15 \%\\              
  SalesProfitability &= Gross Profit / Net Income \cdot 100\% \\
             &= 25313 / 194130 \cdot 100\% = 13 \%\\
 \end{split}
\end{equation}

\subsection{Economic conclusions}
The snore and Sleep Apnea disorder application was analyzed from the economic point of view. It was computed the production cost, different profit and profitability indicators, various types of expenses involved, including direct, indirect, salary and taxes. The whole analysis is worth to understand if the product will be successful and if it's worth investing money in it. The biggest expense represents the intellectual equity, since it is critical to have a reliable product, which is based on extensive research and professional development techniques. The number of expected sold copies might be exaggerated, but it is decided that the product is targeted towards a wider audience. Considering that nowadays most of the individuals own a smartphone, it is easier to get into the market of mobile applications. Deployment is easy, therefore people tend to try different applications from the store. The price of the application can become a blocker, therefore it's price might be dropped. In such scenario other means 
of profit can be introduced, for example in-app advertisement. 

The commercialization of the product is not an easy task. Especially when the product is open sourced. Nevertheless high-quality service and customer support can be provided only to institutions and users that bought the product. The success of the product highly depends on financial strategy and solid economic analysis, which was presented in this chapter.

\clearpage