\section*{Conclusions}
\phantomsection

Obstructive Sleep Apnea and it's direct consequence -- snoring, can be very alarming and serious to one's health. It is important to emphasize the importance and effects of being diagnosed and treated from this disorder. The recovery process involves lifestyle changes and possibly a surgery. Meanwhile there exists a gap between the point where the disorder is present and it being diagnosed at a medical institution. Hence the following simple, fast and flexible piece of software solves this issues by bringing up to one's information that he or she might be having health problems, which were detected via nighttime snoring.

Normally the medical procedure of sleep disorder diagnosis is fairly complicated and involves expensive machinery available in special sleep centers. This is the motivating factor to develop cheap and easy solutions for people who don't have access to medical centers or simply are unaware of their problems.

The outcome of the following thesis is an algorithm that is able to detect snoring with a good enough accuracy. The classification and prediction of snoring sounds offer a fast and scalable solution for users to try in their home environment. One of the reasons of snoring represents Obstructive Sleep Apnea, which isn't easy to detect via acoustic signal dynamics merely. There is no clear evidence or research that will prove the presence of apneas during one's sleep by means of sound. Hence this project became focused on snoring and it's duration and severeness being a reasonable indicator of Obstructive Sleep Apnea disorder. 

In order to become a viable solution for any user, an application for detection and tracking of snoring was developed. The application can run on a variety of Operating Systems, bounded by the availability of Python interpreter on those systems. Therefore the user is not forced to use something very specific and has the possibility to choose. The application gives no guarantee to be 100\% accurate and provides no treatment. Nevertheless it is a non-intrusive way to receive a basic examination over snore and Sleep Apnea problems.

The prototype presented in this thesis is known to run on desktop environments. The target Android platform developed using cross platform framework Kivy proved to be limited in its possibilities related to signal input. Since the project is in constant development and attempts to provide a single interface to all supported platforms, the development of the Audio module is in progress. In the near future the problem might be solved and a better Android support from Kivy can be provided. 

As previously stated there is a lot of work to do in the upcoming future. The learning algorithm can be improved by a richer set of data from different medical centers or even home recordings from people diagnosed with and without Obstructive Sleep Apnea. The more data is collected, the more accurate are going to be the predictions. Moreover a lot of effort can be put into porting the application to mobile platforms. In case Kivy framework fails providing a good solution for signal capturing, the application can be decoupled into two components: processing Web server and the client-side application. The two components will communicate via an API. This will allow keeping the processing part written in Python on the server-side and creating native mobile applications. That being one of the reasons that the application is free and open source, hence available to the public to use, modify and distribute.

Since any piece of software allows improvement, there is always going to be place for enhancements in terms of scientific computations. Nevertheless it was important to have a start, research, analyze and provide a simple solution for the time being. Also it would be nice to provide an update to the user interface, which lacks a designer's touch.

The market of medical appliances and healthcare software solutions available to the major public is a growing area with great potential. The benefits of detecting and tracking Obstructive Sleep Apnea disorder are overwhelming. This is the first step in treatment of the disorder, which has a favorable impact on one's health. The proposed solution aims to be simple and friendly to people who suffer from loud snoring, choking and gasping during sleep, daytime sleepiness and headaches upon awakening. Consequently such an appliance can help people live a healthy and happy life.
\clearpage